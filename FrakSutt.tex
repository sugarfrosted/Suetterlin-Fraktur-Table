\documentclass{article}[12]
\usepackage{amssymb}
\usepackage{graphicx}
\usepackage{suetterl}
\usepackage[T1]{fontenc}
\title{S\"utterlin and Fraktur characters in mathematics.}
\author{James Waddington}
\date{Last updated: \today}
\begin{document}
\maketitle
\section{Introduction} Back in Spring 2014 I audited a course with John Steele at Berkeley. One minor stumbling block I had was his use of S\"utterlin in lieu of Fraktur for models on the board, so I thought I would collect them. If anyone knows of any common usages of Fraktur you should let me know.

\section{Tables of Characters}
\subsection{Common Characters}

\begin{tabular}{c | c | c | l}
    Fraktur & Latin & S\"utterlin & Common Instances\\\hline
    $\mathfrak{A}$ & $A$ &\suetterlin{A} & Models\\
    $\mathfrak{B}$ & $B$ &\suetterlin{B} & Models\\
    $\mathfrak{C}$ & $C$ &\suetterlin{C} & Models\\
    $\mathfrak{H}$ & $H$ &\suetterlin{H} & \\
    $\mathfrak{L}$ & $L$ &\suetterlin{L} & \\
    $\mathfrak{M}$ & $M$ &\suetterlin{M} & Models\\
    $\mathfrak{N}$ & $N$ &\suetterlin{N} & Models (Often models of arithmetic)\\
    $\mathfrak{R}$ & $R$ &\suetterlin{R} & Models (Often of Real Closed Fields), Jacobson Radical\\
    $\mathfrak{a}$ & $a$ &\suetterlin{a} & Ideals\\
    $\mathfrak{b}$ & $b$ &\suetterlin{b} & Ideals\\
    $\mathfrak{c}$ & $c$ &\suetterlin{c} & Ideals, Cardinality of $\mathbb{R}$\\
    $\mathfrak{m}$ & $m$ &\suetterlin{m} & Maximal Ideals\\
    $\mathfrak{p}$ & $p$ &\suetterlin{p} & Prime Ideals\\
\end{tabular}
\end{document}
