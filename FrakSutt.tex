\documentclass{article}[12]
\usepackage{amssymb}
\usepackage{graphicx}
\usepackage{suetterl}
\usepackage[T1]{fontenc}
%\usepackage{showframe}
\title{S\"utterlin and Fraktur Characters in Mathematics.}
\author{James Waddington}
\date{Last updated: \today}
\begin{document}
\maketitle
\section{Introduction} Back in Spring 2014 I audited a course with John Steele at Berkeley. One minor stumbling block I had was his use of S\"utterlin in lieu of Fraktur for models on the board, so I thought I would collect them. If anyone knows of any common usages of Fraktur you should let me know.

\section{Tables of Characters}
\subsection{Common Characters}


\begin{tabular}{c | c | c | p{6cm}}
    Fraktur & Latin & S\"utterlin & Common Instances\\\hline
    $\mathfrak{A}$ & $A$ &\suetterlin{A} & Models\\
    $\mathfrak{B}$ & $B$ &\suetterlin{B} & Models\\
    $\mathfrak{C}$ & $C$ &\suetterlin{C} & Models\\
    $\mathfrak{H}$ & $H$ &\suetterlin{H} & \\
    $\mathfrak{L}$ & $L$ &\suetterlin{L} & \\
    $\mathfrak{M}$ & $M$ &\suetterlin{M} & Models\\
    $\mathfrak{N}$ & $N$ &\suetterlin{N} & Models (Often models of arithmetic)\\
    $\mathfrak{R}$ & $R$ &\suetterlin{R} & Models (Often of Real Closed Fields), Jacobson Radical\\
    $\mathfrak{a}$ & $a$ &\suetterlin{a} & Ideals\\
    $\mathfrak{b}$ & $b$ &\suetterlin{b} & Ideals\\
    $\mathfrak{c}$ & $c$ &\suetterlin{c} & Ideals, Cardinality of $\mathbb{R}$\\
    $\mathfrak{m}$ & $m$ &\suetterlin{m} & Maximal Ideals\\
    $\mathfrak{p}$ & $p$ &\suetterlin{p} & Prime Ideals\\
\end{tabular}

\subsection{Exhaustive List}
\begin{tabular}{c | c | c || c | c | c }
    Fraktur & Latin & S\"utterlin & Fraktur & Latin & S\"utterlin \\\hline
    $\mathfrak{A}$ & $A$ &\suetterlin{A} & 
    $\mathfrak{B}$ & $B$ &\suetterlin{B} \\
    $\mathfrak{C}$ & $C$ &\suetterlin{C} & 
    $\mathfrak{D}$ & $D$ &\suetterlin{D} \\
    $\mathfrak{E}$ & $E$ &\suetterlin{E} & 
    $\mathfrak{F}$ & $F$ &\suetterlin{F} \\
    $\mathfrak{G}$ & $G$ &\suetterlin{G} & 
    $\mathfrak{H}$ & $H$ &\suetterlin{H} \\
    $\mathfrak{I}$ & $I$ &\suetterlin{I} & 
    $\mathfrak{J}$ & $J$ &\suetterlin{J} \\
    $\mathfrak{K}$ & $K$ &\suetterlin{K} & 
    $\mathfrak{L}$ & $L$ &\suetterlin{L} \\ 
    $\mathfrak{M}$ & $M$ &\suetterlin{M} & 
    $\mathfrak{N}$ & $N$ &\suetterlin{N} \\ 
    $\mathfrak{O}$ & $O$ &\suetterlin{O} & 
    $\mathfrak{P}$ & $P$ &\suetterlin{P} \\ 
    $\mathfrak{Q}$ & $Q$ &\suetterlin{Q} & 
    $\mathfrak{R}$ & $R$ &\suetterlin{R} \\ 
    $\mathfrak{S}$ & $S$ &\suetterlin{S} & 
    $\mathfrak{T}$ & $T$ &\suetterlin{T} \\ 
    $\mathfrak{U}$ & $U$ &\suetterlin{U} & 
    $\mathfrak{V}$ & $V$ &\suetterlin{V} \\ 
    $\mathfrak{W}$ & $W$ &\suetterlin{W} & 
    $\mathfrak{X}$ & $X$ &\suetterlin{X} \\ 
    $\mathfrak{Y}$ & $Y$ &\suetterlin{Y} & 
    $\mathfrak{Z}$ & $Z$ &\suetterlin{Z} \\ 

    $\mathfrak{a}$ & $a$ &\suetterlin{a} & 
    $\mathfrak{b}$ & $b$ &\suetterlin{b} \\
    $\mathfrak{c}$ & $c$ &\suetterlin{c} & 
    $\mathfrak{d}$ & $d$ &\suetterlin{d} \\
    $\mathfrak{e}$ & $e$ &\suetterlin{e} & 
    $\mathfrak{f}$ & $f$ &\suetterlin{f} \\
    $\mathfrak{g}$ & $g$ &\suetterlin{g} & 
    $\mathfrak{h}$ & $h$ &\suetterlin{h} \\
    $\mathfrak{i}$ & $i$ &\suetterlin{i} & 
    $\mathfrak{j}$ & $j$ &\suetterlin{j} \\
    $\mathfrak{k}$ & $k$ &\suetterlin{k} & 
    $\mathfrak{l}$ & $l$ &\suetterlin{l} \\ 
    $\mathfrak{m}$ & $m$ &\suetterlin{m} & 
    $\mathfrak{n}$ & $n$ &\suetterlin{n} \\ 
    $\mathfrak{o}$ & $o$ &\suetterlin{o} & 
    $\mathfrak{p}$ & $p$ &\suetterlin{p} \\ 
    $\mathfrak{q}$ & $q$ &\suetterlin{q} & 
    $\mathfrak{r}$ & $r$ &\suetterlin{r} \\ 
    $\mathfrak{s}$ & $s$ &\suetterlin{s} & 
    $\mathfrak{t}$ & $t$ &\suetterlin{t} \\ 
    $\mathfrak{u}$ & $u$ &\suetterlin{u} & 
    $\mathfrak{v}$ & $v$ &\suetterlin{v} \\ 
    $\mathfrak{w}$ & $w$ &\suetterlin{w} & 
    $\mathfrak{x}$ & $x$ &\suetterlin{x} \\ 
    $\mathfrak{y}$ & $y$ &\suetterlin{y} & 
    $\mathfrak{z}$ & $z$ &\suetterlin{z} \\ 
\end{tabular}
\end{document}
